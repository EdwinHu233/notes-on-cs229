\chapter{Generalized Linear Models}
	\section{The Exponential Family}
		如果一种随机分布可以表示为
		\begin{equation}
			p(y; \bm{\eta}) = b(y) \exp \left( \bm{\eta}^\intercal T(y) - \alpha(\bm{\eta}) \right)
		\end{equation}
		则称这种随机分布为以 $ \bm{\eta} $ 为参数的 exponential family。
		\footnote{其中 $ \bm{\eta} $ 称为 natural parameter;$ T(y) $ 称为 sufficient statistic,且通常 $ T(y) = y $;
			$ \alpha(\bm{\eta}) $ 称为 log partition function,起到 normalization 的作用。}
		
		很多常见的随机分布都是 exponential family。
		
		\subsection{Gaussian Distribution} \label{subsec:glm-gau}
			设 $ p(y; \mu) = \frac{1}{\sqrt{2\pi}} \exp \left( - \frac{1}{2} (y-\mu)^2 \right) $,则
			\begin{align*}
				\eta &= \mu \\
				\alpha(\eta) &= \mu^2 / 2 \\
				T(y) &= y \\
				b(y) &= \frac{1}{\sqrt{2\pi}} \exp(-\frac{y^2}{2})
			\end{align*}
			
		\subsection{Bernoulli Distribution} \label{subsec:glm-ber}
			设 $ p(y; \phi) = \phi^y (1-\phi)^{(1-y)} $,则
			\begin{align*}
				\eta &= \log ( \frac{\phi}{1-\phi} ) \\
				\alpha(\eta) &= \log ( 1+\exp(\eta) ) \\
				T(y) &= y \\
				b(y)& = 1
			\end{align*}
			
	\section{GLMs}
		我们把通过以下假设得到的模型称为 \emph{generalized linear models}:
		\begin{itemize}
			\item $ y | \bm{x}; \bm{\theta} \sim \text{ExponentialFamily}(\bm{\eta}) $
			\item $ h_{\bm{\theta}}(\bm{x}) = \mathbb{E}[T(y)|\bm{x};\bm{\theta}] $
			\item $ \bm{\eta} $ 与 $ \bm{x} $ 线性相关,即 $ \bm{\eta} = \bm{\theta}^\intercal \bm{x} $ \footnote{当 $ \bm{\eta} $为向量时,$ \bm{\theta} $ 为矩阵 $ \left[ \bm{\theta}_1, \ldots, \bm{\theta}_n \right] $ }
		\end{itemize}
		GLM 中有三种重要的参数:
		\begin{enumerate}
			\item 随机分布的具体的参数 (如 Bernoulli 中的 $ \phi $,Gaussian 中的 $ \mu $ )
			\item 将随机分布看做是 exponential family 时,natural parameter $ \bm{\eta} $ 
			\item 我们要优化的目标 $ \bm{\theta} $ 
		\end{enumerate}
		这三种参数相互之间存在这样的关系:
		\begin{itemize}
			\item 1 与 2 存在自然的关系,例如 \ref{subsec:glm-ber} 和 \ref{subsec:glm-gau}。
			\item 2 与 3 存在人为设定的关系 $ \bm{\eta} = \bm{\theta}^\intercal \bm{x} $
		\end{itemize}
		这样我们就能求出 1 与 3 的关系。换句话说,优化目标 $ \bm{\theta} $ 与随机分布的具体的参数之间的关系。
		在进行 maximum likelihood 时,将具体的参数替换为 $ \bm{\theta} $,即可求解。
		
		\subsection{Linear Regression}
			假设 $ y|\bm{x};\bm{\theta} \sim \mathcal{N}(\mu, 1^2) $,则由 \ref{subsec:glm-gau}
			\begin{align*}
				h_{\bm{\theta}} (\bm{x}) &= \mathbb{E} [T(y) | \bm{x}; \bm{\theta}] \\
				&= \mu \\
				&= \eta \\
				&= \bm{\theta}^\intercal \bm{x} 
			\end{align*}
			
			Log likelihood 为
			\begin{align*}
				l(\bm{\theta}) &= \sum_{i=1}^{m} \log p(y^{(i)} | \bm{x}^{(i)}; \bm{\theta}) \\
				&= \sum \log \frac{1}{\sqrt{2\pi}} \exp \left( - \frac{1}{2} (y^{(i)}-\mu)^2 \right) \\
				&= -m \cdot \frac{1}{2} (y^{(i)} - \mu)^2 + \text{const} \\
				&= -m \cdot \frac{1}{2} (y^{(i)} - \bm{\theta}^\intercal \bm{x})^2 + \text{const}
			\end{align*}
			对其最大化,可以得到与之前相同的结论。
			
		\subsection{Logistic Regression}
			假设 $ y | \bm{x} ; \bm{\theta} \sim \text{Ber}(\phi) $,则由 \ref{subsec:glm-ber} 
			\begin{align*}
				h_{\bm{\theta}} (\bm{x}) &= \mathbb{E} [T(y) | \bm{x}; \bm{\theta}] \\
				&= \phi \\
				&= \frac{1}{1 + \exp(-\bm{\eta})} \\
				&= \frac{1}{1 + \exp(-\bm{\theta}^\intercal \bm{x})}
			\end{align*}
			
			Log likelihood 为
			\begin{align*}
				l(\bm{\theta}) &= \sum_{i=1}^{m} \log p(y^{(i)} | \bm{x}^{(i)}; \bm{\theta}) \\
				&= \sum \log \phi^{(y^{(i)})} (1-\phi)^{(1-y^{(i)})} \\
				&= \sum y^{(i)} \bm{\theta} \bm{x}^{(i)} - \log \left( 1+\exp(\bm{\theta}^\intercal \bm{x}^{(i)}) \right)
			\end{align*}
			对其最大化,可以得到与之前相同的结论。
			
	\section{Softmax Regression}
		\subsection{Notation}
			对于一个 $ k $ 分类问题,设 $ T(y) \in \mathbb{R}^{k} $ 为 
			\begin{equation*}
				T(1) = 
				\begin{bmatrix}
					1\\
					0\\
					\vdots\\
					0
				\end{bmatrix}
				, \quad T(2) = 
				\begin{bmatrix}
				0\\
				1\\
				\vdots\\
				0
				\end{bmatrix}
				, \ldots, \quad T(k) = 
				\begin{bmatrix}
				0\\
				0\\
				\vdots\\
				1
				\end{bmatrix}
			\end{equation*}
			所以有
			\begin{align}
				T(y)_i &= 1\{y=1\} \\
				\mathbb{E}[T(y)_i] &= \phi_i
			\end{align} 
		
			假设 $ y $ 服从 multinomial distribution
			\footnote{由于 $ \sum_{i=1}^{k} \phi_i = 1 $,所以只有 $ k-1 $ 个独立参数 $ \phi_1, \ldots, \phi_{k-1} $。
				至于 $ \phi_k $ 只是 $ 1-\sum_{i=1}^{k-1} \phi_i $ 的一种简便记法。}
			,即
			 $ p(y = i;\bm{\phi}) = \phi_i $。
			更统一地表示为:
			\begin{equation}
			p(y;\bm{\phi}) = \prod_{i=1}^{k} \phi_i ^{ 1\{ y=i \} } = \prod_{i=1}^{k} \phi_i ^{ T(y)_i } \label{eq:glm-probility1}
			\end{equation}
		
		\subsection{Multinomial Distribution}
			由 \eqref{eq:glm-probility1} 知
			\begin{align*}
				p(y;\bm{\phi}) &= \prod_{i=1}^{k-1} \phi_i ^{ T(y)_i } \cdot \phi_k^{1-\sum_{j=1}^{k-1}T(y)_j} \\
				&= \prod_{i=1}^{k-1} \exp \log \left( \phi_i ^{ T(y)_i } \cdot \phi_k^{1-\sum_{j=1}^{k-1}T(y)_j} \right) \\
				&= \prod_{i=1}^{k-1} \exp \left( \sum_{i=1}^{k-1} \log \left( \frac{\phi_i}{\phi_k} \right) T(y)_i + \log \phi_k  \right)
			\end{align*}
			所以 multinomial distribution 也是 exponential family
			\begin{align}
				\bm{\eta} &= 
				\begin{bmatrix}
					\log(\phi_1 / \phi_k) \\
					\log(\phi_2 / \phi_k) \\
					\vdots \\
					\log(\phi_{k-1} / \phi_k) \\
					\log(\phi_k / \phi_k)
				\end{bmatrix} \label{eq:glm-mult-1} \\
				\alpha(\bm{\eta}) &= - \log(\phi_k) \label{eq:glm-mult-2} \\
				T(y)_i &= 1\{y=1\} \label{eq:glm-mult-3} \\
				b(y) &= 1 \label{eq:glm-mult-4}
			\end{align}
			
			由 \eqref{eq:glm-mult-1} 知
			\begin{align*}
				\eta_i &= \log(\phi_i / \phi_k)  \To \\ 
				\phi_i &= \phi_k \exp(\eta_i) \To \numberthis\label{eq:glm-mult-phi-1} \\
				1 &= \phi_k\sum_{i=1}^{k} \exp(\eta_i) \To \\
				\phi_k &= \frac{1}{\sum_{i=1}^{k} \exp(\eta_i)} \numberthis\label{eq:glm-mult-phi-k}
			\end{align*}
			将 \eqref{eq:glm-mult-phi-k} 代入 \eqref{eq:glm-mult-phi-1} 得
			\begin{equation}
				 \phi_i = \frac{\exp(\eta_i)}{\sum_{j=1}^{k} \exp(\eta_j)} = \frac{\exp(\bm{\theta}_i^\intercal \bm{x})}{\sum_{j=1}^{k} \exp(\bm{\theta}_j^\intercal \bm{x})} \label{eq:glm-mult-phi}
			\end{equation}
			
		\subsection{Solution}
			由 \eqref{eq:glm-probility1} 和 \eqref{eq:glm-mult-phi} 知,log likelihood 为
			\footnote{以下推导中一些细节见\href{https://zhuanlan.zhihu.com/p/24709748}{矩阵求导术}}
			\begin{align*}
				l(\bm{\theta}) &= \sum_{i=1}^{m} \log p(y^{(i)} | \bm{x}^{(i)} ; \bm{\theta} ) \\
				&= \sum_{i=1}^{m} \log \prod_{j=1}^{k} \left( \frac{ \exp(\bm{\theta}_j^\intercal \bm{x}^{(i)}) }{ \sum_{l=1}^{k} \exp \left( \bm{\theta}_l^\intercal \bm{x}^{(i)} \right) } \right)^{1\{y^{(i)}=j\}} \\
				&= \sum_{i=1}^{m} \left[ \sum_{j=1}^{k} 1\{y^{(i)}=j\} \cdot \log \frac {\exp(\bm{\theta}_j^\intercal \bm{x}^{(i)})} {\bm{1}^\intercal \exp(\bm{\theta}^\intercal \bm{x}^{(i)}) } \right] \\
				&= \sum_{i=1}^{m} T(y^{(i)})^\intercal \log
				\begin{bmatrix}
					\frac {\exp(\bm{\theta}_1^\intercal \bm{x}^{(i)})} {\bm{1}^\intercal \exp(\bm{\theta}^\intercal \bm{x}^{(i)}) } \\
					\vdots \\
					\frac {\exp(\bm{\theta}_k^\intercal \bm{x}^{(i)})} {\bm{1}^\intercal \exp(\bm{\theta}^\intercal \bm{x}^{(i)}) }
				\end{bmatrix} \\
				&= \sum_{i=1}^{m} T(y^{(i)}) \log \frac {\exp(\bm{\theta}^\intercal \bm{x}^{(i)})} {\bm{1}^\intercal \exp(\bm{\theta}^\intercal \bm{x}^{(i)}) } 
			\end{align*}
			定义 softmax 函数为
			\begin{equation}
				\softmax(\bm{z}) = \frac {\exp(\bm{z})} {\bm{1}^\intercal \exp(\bm{z}) } 
			\end{equation}
			并设 $ \bm{W} = \bm{\theta}^\intercal $ 则有
			\begin{equation}
				l(\bm{W}) = \sum_{i=1}^{m} T(y^{(i)})^\intercal \log \softmax(\bm{W} \bm{x}^{(i)})
			\end{equation}
			
			设 
			\begin{align*}
				 l' &= T(y)^\intercal \log \softmax(\bm{W} \bm{x}) \\
				 &= T(y)^\intercal \log \exp (\bm{Wx}) - T(y)^\intercal \bm{1} \log (\bm{1}^\intercal \exp \bm{Wx}) \\
				 &= T(y)^\intercal \bm{Wx} - \log \bm{1}^\intercal \exp(\bm{Wx}) 
			\end{align*}
			则
			\begin{align*}
				dl' &= T(y)^\intercal d\bm{W} \bm{x} - \frac{1}{\bm{1}^\intercal \exp(\bm{Wx})} \cdot \bm{1}^\intercal \left( \exp(\bm{Wx}) \odot d\bm{Wx} \right) \\
				&= \tr(\ldots) - \tr(\ldots) \\
				&= \tr(\bm{x} T(y)^\intercal d\bm{W}) - \frac{1}{\bm{1}^\intercal \exp(\bm{Wx})} \tr( (\bm{1} \odot \exp(\bm{Wx}) )^\intercal d\bm{Wx} ) \\
				&= \tr(...) - \tr(\bm{x} \softmax(\bm{Wx})^\intercal d\bm{W} ) \\
				&= \tr(x(T(y) - \softmax(\bm{Wx}))^\intercal d\bm{W})
			\end{align*}
			所以
			\begin{equation}
				\nabla_{\bm{W}} l' = (T(y) - \softmax(\bm{Wx}))^\intercal \bm{x}^\intercal
			\end{equation}
			$ \nabla_{\bm{W}} l $ 的形式也是相似的。
			
			迭代求出 $ \bm{W} $,那么就得到了 hypothesis function 
			\begin{align*}
				h_{\bm{\theta}} (\bm{x})_i &= \mathbb{E} [T(y)_i | \bm{x} ; \bm{\theta}] \\
				&= \mathbb{E} \left[ 1\{y=i\} | \bm{x} ; \bm{\theta} \right] \\
				&= \phi_i \\ 
				&= \frac{\exp(\bm{\theta}_i^\intercal \bm{x})}{\sum_{j=1}^{k} \exp(\bm{\theta}_j^\intercal \bm{x})} \\
			\end{align*}
			即 
			\begin{equation}
				h_{\bm{\theta}} (\bm{x}) = \softmax(\bm{Wx})
			\end{equation}
		
			
			