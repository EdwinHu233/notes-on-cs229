\section{Support Vector Machine}
	SVM 的主要思想是用一个超平面(hyperplane) $ \bm{w}^\intercal \bm{x} + b = 0 $ 分隔样本,使正样本与负样本的\emph{间隔}尽可能大。
	
	\subsection{Notation}
	目标为
	\begin{df}[Hypothesis function]
		\begin{equation*}
			h_{\bm{w},\bm{b}} (\bm{x}) = g( \bm{w}^\intercal \bm{x} + b ) = 
			\begin{cases}
				1 	& \bm{w}^\intercal \bm{x} + b \geq 0 \\
				-1 	& \bm{w}^\intercal \bm{x} + b < 0
			\end{cases}
		\end{equation*}			
	\end{df}
	
	\emph{间隔} 有两种理解
	\begin{df}[Functional margin]
		对于每个样本来说
		\begin{equation*}
			\hat{\gamma}^{(i)} = y^{(i)} (\bm{w^}\intercal \bm{x} + b)
		\end{equation*}
		对于整个样本集来说
		\begin{equation*}
			\hat{\gamma} = \min_{i=1,\ldots,m}{\hat{\gamma}^{(i)}}
		\end{equation*}
	\end{df}
	\begin{df}[Geometric margin]
		对于每个样本来说
		\begin{equation*}
			\gamma = y^{(i)} (\frac{\bm{w^}\intercal}{\| \bm{w} \|} \bm{x} + \frac{b}{\| \bm{w} \|})			
		\end{equation*}
		对于整个样本集来说
			\begin{equation*}
				\gamma = \min_{i=1,\ldots,m}{\gamma^{(i)}}
			\end{equation*}	
	\end{df}
	
	对于正样本,$ y^{(i)} > 0 $,$ \bm{w^}\intercal \bm{x} + b > 0$,反之亦然。(可以通过给 $ \bm{w} \text{和} b$ 同乘 $ -1 $ 来保证)
	
	对于 geometric margin 为 $ 1 $ 的样本,称之为 \emph{支持向量(support vector)}。
	
	\subsection{Lagrange Duality}
	这里对凸优化问题作补充。
	
	设\emph{原优化问题(primal optimization problem)} 为
	\begin{align*}
	\min_w\quad & f(\bm{w}) \\
	\st & \bm{g}(\bm{w}) \leq \bm{0}, \quad \bm{g} \in \mathbb{R}^k \\
	& \bm{h}(\bm{w}) = \bm{0}, \quad \bm{h} \in \mathbb{R}^l
	\end{align*}
	
	\begin{df}[Generalized Lagrangian]
		\begin{equation}
		\mathcal{L}(\bm{w}, \bm{\alpha}, \bm{\beta}) = f(\bm{w}) + \bm{\alpha}^\intercal \bm{g}(\bm{w}) + \bm{\beta}^\intercal \bm{h}(\bm{w})
		\end{equation}
		其中,$ \bm{\alpha} \geq \bm{0} $
	\end{df}
	
	\subsubsection{Primal Optimization Problem}
	定义 $ \theta_\mathcal{P}(\bm{w}) = \max_{\bm{\alpha} \geq \bm{0}, \bm{\beta}}{\mathcal{L}(\bm{w}, \bm{\alpha}, \bm{\beta})} $,显然有
	\begin{equation*}
	\theta_\mathcal{P}(\bm{w}) = 
	\begin{cases}
	f(\bm{w}) & \bm{w} \text{满足约束时} \\
	\infty & \text{其他}
	\end{cases}
	\end{equation*}
	
	所以有
	\begin{equation}
	\min_{\bm{w}}{\theta_\mathcal{P}(\bm{w})} = \min_{\bm{w}}{\max_{\bm{\alpha} \geq \bm{0}, \bm{\beta}}{\mathcal{L}(\bm{w}, \bm{\alpha}, \bm{\beta})}}
	\end{equation}
	就是原优化问题,最优解记为 $ p^* $。
	
	\subsubsection{Dual Optimization Problem}
	定义 $ \theta_\mathcal{D}(\bm{w}) = \min_{\bm{w}}{\mathcal{L}(\bm{w}, \bm{\alpha}, \bm{\beta})} $,则
	\begin{equation}
	\max_{\bm{\alpha} \geq \bm{0}, \bm{\beta}}{\theta_\mathcal{D}(\bm{w})} = \max_{\bm{\alpha} \geq \bm{0}, \bm{\beta}}\min_{\bm{w}}{\mathcal{L}(\bm{w}, \bm{\alpha}, \bm{\beta})} \label{dual-opt-problem}
	\end{equation}
	称为对偶问题,最优解记为 $ d^* $。
	
	\subsubsection{KKT Condition}\label{sec-kkt}
	显然 $ d^* \leq p^* $,并且在一定条件下,$ d^* = p^* $。一种常见的条件是:
	\begin{itemize}
		\item $ f $ 和 $ g_i $ 均为凸函数\footnote{凸函数(convex function):Hessian 矩阵半正定}
		\item $ h_i $ 均为仿射函数\footnote{仿射函数(affine function):形如 $ f(\bm{x}) = \bm{a}^\intercal \bm{x} + b $}
		\item $ g $ 严格可满足\footnote{严格可满足(strictly feasible):$ \exists \bm{w}: g(\bm{w}) \leq \bm{0} $}
	\end{itemize}
	
	此时,设 $ p^* = d^* = \mathcal{L}(\bm{w}^*, \bm{\alpha}^*, \bm{\beta}^*) $,则 $ \bm{w}^*, \bm{\alpha}^*, \bm{\beta}^* $ 满足
	\begin{align}
	\frac{\partial}{\partial \bm{w}} \mathcal{L}(\bm{w}^*, \bm{\alpha}^*, \bm{\beta}^*) &= \bm{0}\label{kkt1} \\
	\frac{\partial}{\partial \bm{\beta}} \mathcal{L}(\bm{w}^*, \bm{\alpha}^*, \bm{\beta}^*) &= \bm{0} \\
	\alpha_i^* g_i(\bm{w}^*) &=0, \quad i = 1,\ldots,k \label{kkt3}\\
	g(\bm{w}^*) & \leq \bm{0} \\
	\bm{\alpha}^* & \geq \bm{0}
	\end{align}
	以上方程组称为 $ \bm{w}^*, \bm{\alpha}^*, \bm{\beta}^* $ 的 KKT 条件
	
	注意 \eqref{kkt3} 说明了 $ g(\bm{w}^*) < 0 \to \alpha_i = 0 $。
	特别地,在 SVM 中只有支持向量对应的 $ \alpha_i $ 可能非零。
	
	\subsection{Optimal Margin Classifier}
		\subsubsection{Primal Problem}
		假定样本集线性可分,那么有优化问题
		\begin{align*}
			\max_{\gamma, \bm{w}, b}\quad & \frac{\hat{\gamma}}{\| \bm{w} \|} \\
			\st & y^{(i)} (\bm{w^}\intercal \bm{x} + b) \geq \hat{\gamma}, \quad i = 1, \ldots, m
		\end{align*}
		求解此优化问题,则得到 \emph{Optimal margin classifier}。
		
		通过同时对 $ \bm{w} $ 和 $ b $ 放缩,可以使 $ \hat{\gamma} = 1 $ 。再做变形可以得到等价的优化问题
		\begin{align}
			\min_{\gamma, \bm{w}, b}\quad & \frac{1}{2} \| \bm{w} \| ^2 \\
			\st & y^{(i)} (\bm{w^}\intercal \bm{x} + b) \geq 1, \quad i = 1, \ldots, m
		\end{align}
		约束条件为 $ g_i(\bm{w}) = -y^{(i)} (\bm{w^}\intercal \bm{x} + b) + 1 \leq 0 $。
		
		\subsubsection{Dual Problem}
		此问题的 generalized Lagrangian 为
		\begin{equation}
			\mathcal{L}(\bm{w}, b, \bm{\alpha}) = \frac{1}{2} \| \bm{w} \| ^2 - \sum_{i=1}^m \alpha_i \left[ y^{(i)} (\bm{w^}\intercal \bm{x}^{(i)} + b) - 1 \right] \label{svm-lag}
		\end{equation}
		
		可以使用 KKT 条件的 \eqref{kkt1} 得到
		\begin{align}
			&\bm{w}^* = \sum_i{\alpha_i y^{(i)} \bm{x}^{(i)}} \label{svm-kkt1-1} \\ 
			&\sum_i{\alpha_i y^{(i)}} = 0 \label{svm-kkt1-2}
		\end{align}
		带入到 \eqref{svm-lag} 中,得到
		\begin{equation*}
			\mathcal{L}(\bm{w}^*, b^*, \bm{\alpha}) = \sum_i{\alpha_i} - \frac{1}{2} \sum_i \sum_j {\alpha_i \alpha_j y^{(i)} y^{(j)} \left\langle \bm{x}^{(i)} \bm{x}^{(j)} \right\rangle }
		\end{equation*}
		
		所以,对偶问题为
		\begin{align}
			\max_{\bm{\alpha}} \quad & W(\bm{\alpha}) = \sum_{i=1}{\alpha_i} - \frac{1}{2} \sum_i \sum_j {\alpha_i \alpha_j y^{(i)} y^{(j)} \left\langle \bm{x}^{(i)} \bm{x}^{(j)} \right\rangle } \\
			\st & \bm{\alpha} \geq \bm{0} \\
			& \sum_{i=1}^{m} \alpha_i y^{(i)} = 0
		\end{align}
		注意\eqref{dual-opt-problem} 的形式,$ \max \ldots $ 得到了第一个约束条件,
		$ \min \ldots $ 得到了第二个约束条件(以及被代入 generalized Lagrangian 的 \eqref{svm-kkt1-1})。
		
		\subsubsection{Solution}
		利用之后的方法对对偶问题求解,得到 $ \bm{\alpha}^* $。代入 \eqref{svm-kkt1-1} 可以解得 $ \bm{w}^* $。
		
		根据 $ \bm{w}^* $ 可以解得
		\begin{equation}
			b^* = - \frac{1}{2} \left( \max_{i: y^{(i)} = -1} (\bm{w}^*)^\intercal \bm{x}^{(i)} + \min_{i: y^{(i)} = 1} (\bm{w}^*)^\intercal \bm{x}^{(i)} \right) \label{svm-commonly-solving-b}
		\end{equation}
		\begin{pf}
			对于支持向量,有
			\begin{align*}
				\max_{i: y^{(i)} = -1} (\bm{w}^*)^\intercal \bm{x}^{(i)} + b & = -1 \\
				\min_{i: y^{(i)} = 1} (\bm{w}^*)^\intercal \bm{x}^{(i)} + b & = 1
			\end{align*}
			两式相加即可。
		\end{pf}
		
		解出参数 $ \bm{w} $ 和 $ b $ 之后,就可以对于输入 $ \bm{x} $ 输出预测值 $ y $ 了。
		由 \eqref{svm-kkt1-1} 得
		\begin{align}
			\bm{w}^\intercal \bm{x} + b &= \left( \sum_i \alpha_i y^{(i)} \bm{x}^{(i)} \right)^\intercal \bm{x} + b \\
			&= \sum_i \alpha_i y^{(i)} \left\langle \bm{x}^{(i)}, \bm{x} \right\rangle + b
		\end{align}
		对于此式,注意两点:
		\begin{itemize}
			\item 由 \ref{sec-kkt} 中的结论,只有支持向量对应的 $ \alpha_i $ 可能非零。支持向量的个数非常少,故预测的代价很小
			\item 此式整理为了内积的形式,内积可以用核函数代替(对偶问题也一样)
		\end{itemize}
		
	\subsection{Kernels}
	\begin{df}[Feature mapping]
		我们称原始输入的数据为样本的 attributes,对 attributes 映射得到 features。这样的映射称为 feature mapping,记作 $ \phi(\bm{x}) $
	\end{df}
	例如,设房价是一个标量 $ x $,我们可以映射得到 $ \phi(x) = {\left( x, x^2, x^3  \right)} ^\intercal $
	
	\begin{df}[Kernel method]
		\begin{equation*}
			K(\bm{x}, \bm{z}) = \phi(\bm{x})^\intercal \phi(\bm{z})
		\end{equation*}
	\end{df}
	我们可以用“核函数”来代替向量内积。
	
	直觉上,内积表示了两个向量的“相似”程度:越是接近平行的向量,内积的绝对值越大;越是接近正交的向量,内积的绝对值越小。
	Kernel methods 也应该具有类似的特征。
	
		\subsubsection{Common kernels}
		常见的 kernel 有:
		
		\begin{df}[Polynomial kernel]
			\begin{equation}
				K(\bm{x}, \bm{z}) = ( \bm{x}^\intercal \bm{z} + c )^d
			\end{equation}
		\end{df}
		这个 kernel 将 $ n $ 维向量映射到 $ \binom{n+d}{d} $ 的 feature space 中。换言之,将原本 $ O(n^d) $ 的复杂度降低到 $ O(n) $。
		
		\begin{df}[Gaussian kernel]
			\begin{equation}
				K(\bm{x}, \bm{z}) = \exp\left( - \frac{ \| \bm{x} - \bm{z} \|^2 }{2 \sigma^2} \right)
			\end{equation}
		\end{df}
		这个 kernel 将 $ n $ 维向量映射到无穷维的 feature space 中。复杂度为 $ O(n) $
		
		\subsubsection{Validation}
		\begin{df}[Kernel matrix]
			设 $ \{ \bm{x}^{(1)}, \ldots, \bm{x}^{(m)} \} $ 为数据集,$ K $ 是一个合法的 kernel(即 $ \exists \phi: \ldots $ ),
			则对应的 kernel matrix 定义为
			\begin{equation*}
				K_{i,j} = K(\bm{x}^i, \bm{x}^j)
			\end{equation*}
		\end{df}
		
		可证 $ K $ 是对称半正定的
		\begin{pf}
			显然 $ K $ 是对称的。接下来只需证 $ \forall \bm{z}: \bm{z}^\intercal K \bm{z} \geq 0 $ 即可:
			\begin{align*}
				\bm{z}^\intercal K \bm{z} &= \sum_i \sum_j z_i K_{ij} z_j \\
				&= \sum_i \sum_j z_i \phi(\bm{x}^{(i)})^\intercal \phi(\bm{x}^{(j)}) z_j \\
				&= \sum_i \sum_j \sum_k z_i \phi_k(\bm{x}^{(i)}) \phi_k(\bm{x}^{(j)}) z_j \\
				&= \sum_k \sum_i \sum_j (z_i \phi_k(\bm{x}^{(i)})) (z_j \phi_k(\bm{x}^{(j)})) \\
				&= \sum_k \sum_i ( z_i \phi_k(\bm{x}^{(i)}) )^2 \\
				&\geq 0
			\end{align*}
		\end{pf}
		
		事实上,若 $ K $ 是对称半正定的,则反过来也可证 $ K $ 是一个合法的 kernel。即以下定理:
		\begin{thm}
			设 $ K: \mathbb{R}^n \times \mathbb{R}^n \to \mathbb{R} $,$ \{ \bm{x}^{(1)}, \ldots, \bm{x}^{(m)} \} $ 为 $ \mathbb{R}^n $ 上的数据集,则
			\[ K \text{对称半正定} \iff K \text{是合法的 kernel} \]
		\end{thm}
		
	\subsection{$ l1 $ Norm Soft Margin SVM}
		\subsubsection{Primal Problem}
		有时数据集不是严格线性可分的,而是存在个别异常的点。我们可以允许部分点的 $ \gamma^{(i)} < 1 $ (甚至为负)。
		一种方法是引入 $ l1 $ regularization。
		\begin{align*}
			\min_{\bm{w}, b, \bm{\xi}} \quad & \frac{1}{2} \| \bm{w} \|^2 + C \sum_i \xi_i \\
			\st & y^{(i)} (\bm{w^}\intercal \bm{x} + b) \geq 1 - \xi_i, \quad i = 1, \ldots, m \\
			& \xi_i \geq 0, \quad i = 1, \ldots, m
		\end{align*}
		其中 $ \xi_i $ 为允许的误差。($ \bm{w}, b, \bm{xi} $ 都是自变量)
		
		\subsubsection{Dual Problem}
		Generalized Lagrangian 为
		\begin{equation*}
			\mathcal{L}( \bm{w}, b, \bm{xi}, \bm{\alpha}, \bm{r} ) = \frac{1}{2} \| \bm{w} \|^2 + C \sum_i \xi_i - \sum_i \alpha_i \left[ y^{(i)} (\bm{w^}\intercal \bm{x} + b) - 1 + \xi_i \right] - \sum_i r_i \xi_i
		\end{equation*}
		类似于普通的 SVM,由 \eqref{kkt1} 得
		\begin{align}
			\sum_i \alpha_i y^{(i)} \bm{x}^{(i)} &= \bm{w}^* \\
			\sum_i \alpha_i y^{(i)} &= 0 \\
			\alpha_i + r_i &= C
		\end{align}
		代入上式得
		\begin{align}
			\max_{\bm{\alpha}} \quad & W(\bm{\alpha}) = \sum_i \alpha_i - \frac{1}{2} \sum_i \sum_j \alpha_i \alpha_j y^{(i)} y^{(j)} \left\langle \bm{x}^{(i)} , \bm{x}^{(j)} \right\rangle \\
			\st & 0 \leq \alpha_i \leq C \\
			& \sum_i \alpha_i y^{(i)} = 0
		\end{align}
		
		\subsubsection{Solution}
		与普通的 SVM 类似,求解对偶问题得到 $ \bm{\alpha}^* $ 后,即可求得 $ \bm{w}^* $。
		与 \eqref{svm-commonly-solving-b} 不同的是,此时支持向量不再是 functional margin 最小的点,故 $ b^* $ 的求解不再适用此式。
		%TODO
		
		\subsubsection{KKT Dual-complimentary Conditions}
		对 KKT 条件稍作变形可得
		\begin{align*}
			\alpha_i \geq 0, \quad & r_i \geq 0, \quad \alpha_i + r_i = C \\
			\xi_i \geq 0, \quad & y^{(i)} ( \bm{w}^\intercal \bm{x}^{(i)} + b ) \geq 1 - \xi_i \\
			r_i \xi_i = 0, \quad & \alpha \left( \bm{w}^\intercal \bm{x}^{(i)} + b - (1-\xi_i) \right) = 0
		\end{align*}
		易得
		\begin{align}
			\alpha_i = 0 & \To y^{(i)} ( \bm{w}^\intercal \bm{x}^{(i)} + b ) \geq 1 \label{eq:svm-soft-margin-kkt1} \\ 
			\alpha_i = C & \To y^{(i)} ( \bm{w}^\intercal \bm{x}^{(i)} + b ) \leq 1 \label{eq:svm-soft-margin-kkt2} \\
			0 \leq \alpha_i \leq C & \To y^{(i)} ( \bm{w}^\intercal \bm{x}^{(i)} + b ) = 1\label{eq:svm-soft-margin-kkt3}		
		\end{align}
		
		其中 \eqref{eq:svm-soft-margin-kkt1} 是普通样本(或支持向量)的情况,\eqref{eq:svm-soft-margin-kkt2} 是异常样本(或支持向量)的情况,
		\eqref{eq:svm-soft-margin-kkt3} 是支持向量的情况。