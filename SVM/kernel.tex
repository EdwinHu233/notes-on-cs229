\section{Kernels}
	\begin{df}[Feature mapping]
		我们称原始输入的数据为样本的 attributes,对 attributes 映射得到 features。这样的映射称为 feature mapping,记作 $ \phi(\bm{x}) $
	\end{df}
	例如,设房价是一个标量 $ x $,我们可以映射得到 $ \phi(x) = {\left( x, x^2, x^3  \right)} ^\intercal $
	
	\begin{df}[Kernel method]
		\begin{equation*}
			K(\bm{x}, \bm{z}) = \phi(\bm{x})^\intercal \phi(\bm{z})
		\end{equation*}
	\end{df}
	我们可以用“核函数”来代替向量内积。
	
	直觉上,内积表示了两个向量的“相似”程度:越是接近平行的向量,内积的绝对值越大;越是接近正交的向量,内积的绝对值越小。
	Kernel methods 也应该具有类似的特征。
	
	\subsection{Common kernels}
		常见的 kernel 有:
		
		\begin{df}[Polynomial kernel]
			\begin{equation}
				K(\bm{x}, \bm{z}) = ( \bm{x}^\intercal \bm{z} + c )^d
			\end{equation}
		\end{df}
		这个 kernel 将 $ n $ 维向量映射到 $ \binom{n+d}{d} $ 的 feature space 中。换言之,将原本 $ O(n^d) $ 的复杂度降低到 $ O(n) $。
		
		\begin{df}[Gaussian kernel]
			\begin{equation}
				K(\bm{x}, \bm{z}) = \exp\left( - \frac{ \| \bm{x} - \bm{z} \|^2 }{2 \sigma^2} \right)
			\end{equation}
		\end{df}
		这个 kernel 将 $ n $ 维向量映射到无穷维的 feature space 中。复杂度为 $ O(n) $
		
	\subsection{Validation}
		\begin{df}[Kernel matrix]
			设 $ \{ \bm{x}^{(1)}, \ldots, \bm{x}^{(m)} \} $ 为数据集,$ K $ 是一个合法的 kernel(即 $ \exists \phi: \ldots $ ),
			则对应的 kernel matrix 定义为
			\begin{equation*}
				K_{i,j} = K(\bm{x}^i, \bm{x}^j)
			\end{equation*}
		\end{df}
		
		可证 $ K $ 是对称半正定的
		\begin{proof}
			显然 $ K $ 是对称的。接下来只需证 $ \forall \bm{z}: \bm{z}^\intercal K \bm{z} \geq 0 $ 即可:
			\begin{align*}
				\bm{z}^\intercal K \bm{z} &= \sum_i \sum_j z_i K_{ij} z_j \\
				&= \sum_i \sum_j z_i \phi(\bm{x}^{(i)})^\intercal \phi(\bm{x}^{(j)}) z_j \\
				&= \sum_i \sum_j \sum_k z_i \phi_k(\bm{x}^{(i)}) \phi_k(\bm{x}^{(j)}) z_j \\
				&= \sum_k \sum_i \sum_j (z_i \phi_k(\bm{x}^{(i)})) (z_j \phi_k(\bm{x}^{(j)})) \\
				&= \sum_k \sum_i ( z_i \phi_k(\bm{x}^{(i)}) )^2 \\
				&\geq 0
			\end{align*}
		\end{proof}
		
		事实上,若 $ K $ 是对称半正定的,则反过来也可证 $ K $ 是一个合法的 kernel。即以下定理:
		\begin{thm}
			设 $ K: \mathbb{R}^n \times \mathbb{R}^n \to \mathbb{R} $,$ \{ \bm{x}^{(1)}, \ldots, \bm{x}^{(m)} \} $ 为 $ \mathbb{R}^n $ 上的数据集,则
			\[ K \text{对称半正定} \iff K \text{是合法的 kernel} \]
		\end{thm}
		
