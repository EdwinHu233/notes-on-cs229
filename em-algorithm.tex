\chapter{The EM Algorithm}
	\section{Jensen's Inequality}
		设 $ f $ 是一个凸函数,$ \bm{X} $ 是一个随机变量,则:
		\begin{equation}
			\mathbb{E}[f(\bm{X})] \geq f(\mathbb{E}[\bm{X}]) \label{eq:jeson-ineq}
		\end{equation}
		特别地,当 $ f $ 是严格凸函数时,有:
		\begin{equation}
			\mathbb{E}[f(\bm{X})] = f(\mathbb{E}[\bm{X}]) \iff p(\bm{X} = \mathbb{E}[\bm{X}]) = 1 \label{eq:jeson-ineq-extra}
		\end{equation}
		
		若 $ f $ 是凹函数,则有相反的结论。
		
	\section{The EM Algorithm}
		对于数据集 $ \{ \bm{x}^{(1)}, \ldots, \bm{x}^{(m)} \} $,若要用分布 $ p(\bm{x}, z) $ 去拟合,则:
		\begin{align*}
			l(\bm{\theta}) &= \sum_{i=1}^{m} \log p(\bm{x}^{(i)}; \bm{\theta}) \\
			&= \sum_{i=1}^{m} \log \sum_{z^{(i)}} p(\bm{x}^{(i)}, z^{(i)}; \bm{\theta}) \\
			&= \sum_{i=1}^{m} \log \sum_{z^{(i)}} Q_i(z^{(i)}) \frac{p(\bm{x}^{(i)}, z^{(i)}; \bm{\theta})}{Q_i(z^{(i)})}
		\end{align*}
		我们希望 $ \forall i, \sum_{z} Q_i(z) = 1 \quad \land \quad \forall i, z, Q_i(z) \geq 0 $。
		这样就可以把 $ Q_i $ 看做一个随机分布,$ z^{(i)} $ 是这个随机分布产生的样本。
		接上式,
		\begin{equation*}
			l(\bm{\theta}) = \sum_i \log \mathbb{E}_{z^{(i)} \sim Q_i} \left[ \frac{p(\bm{x}^{(i)}, z^{(i)}; \bm{\theta})}{Q_i(z^{(i)})} \right] 
		\end{equation*}
		
		\subsection{Lower Bound}
			由 \eqref{eq:jeson-ineq} 的凹函数版本知,$ \frac{p(\bm{x}^{(i)}, z^{(i)}; \bm{\theta})}{Q_i(z^{(i)})} $ 相当于随机变量 $ X $,$ \log $ 相当于凹函数 $ f $,则
			\begin{align*}
				l(\bm{\theta}) &= \sum_i f(\mathbb{E}[\bm{X}]) \\
				&\geq \sum_i \mathbb{E}[f(\bm{X})] \\
				&= \sum_i \sum_{z^{(i)}} Q_i(z^{(i)}) \log \frac{p(\bm{x}^{(i)}, z^{(i)}; \bm{\theta})}{Q_i(z^{(i)})} \label{eq:em-algo-lower-bound} \numberthis
			\end{align*}
			
			对于任意给定的 $ \bm{\theta} $, \eqref{eq:em-algo-lower-bound} 给出了 $ l $ 的下界。
			当改变 $ \bm{\theta} $ 使这个下界上升时,$ l $ 也上升。
		
		\subsection{The Best Lower Bound}
			我们希望,对于当前的 $ \bm{\theta} $,这个下界是 $ l $ 的下确界;
			之后改变 $ \bm{\theta} $ 使这个下界最大化,以此优化 $ l $。
			这就取决于如何选择 $ Q_i $ 了。
			
			由 \eqref{eq:jeson-ineq-extra} 知,对于当前的 $ \bm{\theta} $, 若
			\begin{equation*}
				\forall i, \exists c \in \mathbb{R}, \qquad \frac{p(\bm{x}^{(i)}, z^{(i)}; \bm{\theta})}{Q_i(z^{(i)})} = c
			\end{equation*}
			那么 $ X \text{是常数} \Rightarrow p({X} = \mathbb{E}[{X}]) = 1 \Rightarrow \mathbb{E}[f({X})] = f(\mathbb{E}[{X}]) $,
			这就确保了在当前 $ \bm{\theta} $ 下界是下确界。
			
			不妨取 $ Q_i(z^{(i)}) \propto p(\bm{x}^{(i)}, z^{(i)}; \bm{\theta}) $,又由 $ \sum_{z^{(i)}} Q_i(z^{(i)}) = 1 $ 知,
			\begin{align*}
				Q_i(z^{(i)}) &= \frac{p(\bm{x}^{(i)}, z^{(i)}; \bm{\theta})}{\sum_z p(\bm{x}^{(i)}, z; \bm{\theta})} \\
				&= p(z^{(i)} | \bm{x}^{(i)}; \bm{\theta}) \label{eq:em-algo-posterior} \numberthis
			\end{align*}
			综上,只需取 $ Q_i $ 为后验概率 \eqref{eq:em-algo-posterior} 即可。
		
		\subsection{summary}
			根据 \eqref{eq:em-algo-lower-bound} 和 \eqref{eq:em-algo-posterior}。EM 算法为:
			
			迭代直到收敛:
			\begin{align}
				\text{(E-step)} \qquad\qquad\qquad\qquad& \\
				 \forall i, \quad Q_i(z^{(i)}) &= p(z^{(i)} | \bm{x}^{(i)}; \bm{\theta}) \\
				\text{(M-step)} \qquad\qquad\qquad\qquad& \\
				 \bm{\theta} &= \argmax_{\bm{\theta}} \sum_i \sum_{z^{(i)}} Q_i(z^{(i)}) \log \frac{p(\bm{x}^{(i)}, z^{(i)}; \bm{\theta})}{Q_i(z^{(i)})}
			\end{align}