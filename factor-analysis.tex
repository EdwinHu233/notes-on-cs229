\chapter{Factor Analysis}
	之前的 mixtures of Gaussian 适用于 $ m > n $ 的情况。
	而当 $ m \ll n $ 时,即使只用一个 Gaussian distribution 去拟合,得到的 $ \bm{\Sigma} $ 也是奇异矩阵
	\footnote{因为 $ m \ll n $,所以 $ \{ \bm{x}^{(1)}, \ldots, \bm{x}^{(m)} \} $ 只能 span 出 $ \mathbb{R}^n $ 的一个维度很小的 subspace }。
	
	一种思路是,对 $ \bm{\Sigma} $ 进行一定的限制,强行使 $ \bm{\Sigma} $ 对角化。但这样会丢失各个特征之间的联系,效果并不好。
	另一种思路是,假设高维的样本 $ \bm{x}^{(i)} \in \mathbb{R}^n $ 由某低维数据 $ \bm{z}^{(i)} \in \mathbb{R}^k $ 仿射得到。
	
	\section{Marginals \& Conditionals Of Gaussians}
		假设
		\footnote{其中 $ \bm{x}_1 \in \mathbb{R}^r, \bm{x}_2 \in \mathbb{R}^s; \quad \bm{\mu}_1 \in \mathbb{R}^r, \bm{\mu}_2 \in \mathbb{R}^s; \quad \bm{\Sigma}_{11} \in \mathbb{R}^{r \times r}, \bm{\Sigma}_{12} \in \mathbb{R}^{r \times s}, \bm{\Sigma}_{21} \in \mathbb{R}^{s \times r}, \bm{\Sigma}_{22} \in \mathbb{R}^{s \times s} $,且 $ \bm{\Sigma}_{12} = \bm{\Sigma}_{21}^\intercal $。}
		\begin{equation*}
			\bm{x} = 
			\begin{bmatrix}
				\bm{x}_1 \\
				\bm{x}_2
			\end{bmatrix}
			\sim \mathcal{N}(\bm{\mu}, \bm{\Sigma}), \qquad \bm{\mu} = 
			\begin{bmatrix}
				\bm{\mu}_1 \\
				\bm{\mu}_2
			\end{bmatrix}
			, \bm{\Sigma} = 
			\begin{bmatrix}
				\bm{\Sigma}_{11} & \bm{\Sigma}_{12} \\
				\bm{\Sigma}_{21} & \bm{\Sigma}_{22} 
			\end{bmatrix}
		\end{equation*} 
		
		可得 $ \bm{x}_1 \sim \mathcal{N}(\bm{\mu}_1, \bm{\Sigma}_{11}), \quad \bm{x}_1 | \bm{x}_2 \sim \mathcal{N}(\bm{\mu}_{1|2}, \bm{\Sigma}_{1|2}) $,其中 
		\begin{align}
			\bm{\mu}_{1|2} &= \bm{\mu}_1 + \bm{\Sigma}_{12} \bm{\Sigma}_{22}^{-1} (\bm{x}_2 - \bm{\mu}_2) \\
			\bm{\Sigma}_{1|2} &= \bm{\Sigma}_{11} - \bm{\Sigma}_{12} \bm{\Sigma}_{22}^{-1} \bm{\Sigma}_{21}
		\end{align}
		
	\section{The Factor Analysis Model}	
		假设 
		\begin{align*}
			\bm{z} &\sim \mathcal{N}(\bm{0}, \bm{I}) \\
			\bm{x} &\sim \mathcal{N}(\bm{\mu} + \bm{\Lambda z}, \bm{\Psi})
		\end{align*}
		其中 $ \bm{z} \in \mathbb{R}^k $ 是一个低维的隐含随机变量,经过仿射后产生 $ \bm{x} $;$ \bm{\Lambda} \in \mathbb{R}^{n \times k} $;
		$ \bm{\Psi} \in \mathbb{R}^{n \times n} $ 是一个对角矩阵。
		
		一种等价的说法是:
		\begin{align}
			\bm{z} &\sim \mathcal{N}(\bm{0}, \bm{I}) \\
			\bm{\epsilon} &\sim \mathcal{N}(\bm{0}, \bm{\Psi}) \\
			\bm{x} &= \bm{\mu} + \bm{\Lambda z} + \bm{\epsilon}
		\end{align}
		
		设 
		\begin{equation*}
			\begin{bmatrix}
				\bm{z} \\
				\bm{x} 
			\end{bmatrix}
			\sim \mathcal{N}(\bm{\mu}_{zx}, \bm{\Sigma})
		\end{equation*}
		易知 $ \mathbb{E}[\bm{z}] = 0, \quad \mathbb{E}[\bm{x}] = \mathbb{E}[\bm{\mu} + \bm{\Lambda z} + \bm{\epsilon}] = \bm{\mu} $,所以
		\begin{equation*}
			\bm{\mu}_{zx} = 
			\begin{bmatrix}
				\bm{0} \\
				\bm{\mu}
			\end{bmatrix}
		\end{equation*}
		又设
		\begin{align*}
			\bm{\Sigma} &= \mathbb{E} 
			\left[  
				\left( 
					\begin{bmatrix}
						\bm{z} \\
						\bm{x}
					\end{bmatrix}
					- \mathbb{E} 
					\begin{bmatrix}
						\bm{z} \\
						\bm{x}
					\end{bmatrix}
				\right)
				\left( 
					\begin{bmatrix}
						\bm{z} \\
						\bm{x}
					\end{bmatrix}
					- \mathbb{E} 
					\begin{bmatrix}
						\bm{z} \\
						\bm{x}
					\end{bmatrix}
				\right)^\intercal
			\right] \\
			&= \begin{bmatrix}
				\mathbb{E}\left[ (\bm{z} - \mathbb{E}\bm{z}) (\bm{z} - \mathbb{E}\bm{z})^\intercal \right]  & \mathbb{E}\left[ (\bm{z} - \mathbb{E}\bm{z}) (\bm{x} - \mathbb{E}\bm{x})^\intercal \right] \\
				\mathbb{E}\left[ (\bm{x} - \mathbb{E}\bm{x}) (\bm{z} - \mathbb{E}\bm{z})^\intercal \right] & \mathbb{E}\left[ (\bm{x} - \mathbb{E}\bm{x}) (\bm{x} - \mathbb{E}\bm{x})^\intercal \right]
			\end{bmatrix} \\
			&= \begin{bmatrix}
				\bm{\Sigma}_{zz} & \bm{\Sigma}_{zx} \\
				\bm{\Sigma}_{xz} & \bm{\Sigma}_{xx}
			\end{bmatrix}
		\end{align*}
		其中 
		\begin{align*}
			\bm{\Sigma}_{zz} &= \bm{I} \\
			\bm{\Sigma}_{zx} &= \mathbb{E}[\bm{z} \cdot (\bm{\mu} + \bm{\Lambda z} + \bm{\epsilon} - \bm{\mu})] \\
			&= \mathbb{E}[ \bm{z}(\bm{\Lambda z})^\intercal + \bm{z}\bm{\epsilon}^\intercal ] \\
			&= \mathbb{E}[\bm{z z}^\intercal] \bm{\Lambda}^\intercal + \mathbb{E}[\bm{z}] \mathbb{E}[\bm{\epsilon}]^\intercal \\
			&= \bm{\Lambda}^\intercal \\
			\bm{\Sigma}_{xz} &= \bm{\Sigma}_{zx}^\intercal \\
			&= \bm{\Lambda} \\
			\bm{\Sigma}_{xx} &= \mathbb{E}[(\bm{\mu} + \bm{\Lambda z} + \bm{\epsilon} - \bm{\mu}) (\bm{\mu} + \bm{\Lambda z} + \bm{\epsilon} - \bm{\mu})^\intercal] \\
			&= \bm{\Lambda} \bm{\Lambda}^\intercal + \bm{\Psi}
		\end{align*}
		
		综上所述,
		\begin{equation}
			\begin{bmatrix}
				\bm{z} \\
				\bm{x}
			\end{bmatrix} 
			\sim \mathcal{N}\left( 
			\begin{bmatrix}
				\bm{0} \\
				\bm{\mu}
			\end{bmatrix}, 
			\begin{bmatrix}
				\bm{I} & \bm{\Lambda}^\intercal \\
				\bm{\Lambda} & \bm{\Lambda} \bm{\Lambda}^\intercal + \bm{\Psi}
			\end{bmatrix}
			\right)
		\end{equation}