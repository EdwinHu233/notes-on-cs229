\chapter{Generative Leaning Algorithms}
	之前学习的算法都是 \emph{discriminative learning algorithm},下面介绍另一种算法 \emph{generative learning algorithm}。
	二者有以下区别:
	\begin{itemize}
		\item Discriminative : 直接对 $ p(y | \bm{x}; \bm{\theta}) $ 建模,得到预测值
		\item Generative : 对于不同的标签值 $ y_1, y_2, ... $ ,对 $ p(\bm{x} | y_1), p(\bm{x} | y_2), \ldots $ 建模,
		再根据 Bayes 公式得到 $ \argmax_y p(y | \bm{x}) $ 作为预测值。
	\end{itemize}
	
	\section{Gaussian Discriminant Analysis}
		\subsection{Multivariate Gaussian Distribution}
			概率分布如下
			\begin{equation}
				p(\bm{x}; \bm{\mu}, \bm{\Sigma} ) = \frac{1}{(2\pi)^{n/2} |\bm{\Sigma}|^{1/2}} \exp \left( -\frac{1}{2} (\bm{x}-\bm{\mu})^\intercal \bm{\Sigma}^{-1} (\bm{x}-\bm{\mu}) \right)
			\end{equation}
			其中 $ \bm{\mu} = \mathbb{E}[\bm{X}] .\quad \bm{\Sigma} = \text{Cov}(\bm{X}) = \mathbb{E}\left[ (\bm{X}-\mathbb{E}[\bm{X}]) (\bm{X}-\mathbb{E}[\bm{X}])^\intercal \right] $
			
		\subsection{GDA Model} 
			对一个二分类问题,作如下假设
			\footnote{这里假设两个 Gaussian distribution 的协方差矩阵相同}
			\begin{align*}
				y &\sim \text{Ber}(\phi) \\
				\bm{x} | y=0 &\sim \mathcal{N} (\bm{\mu}_0, \bm{\Sigma}) \\
				\bm{x} | y=1 &\sim \mathcal{N} (\bm{\mu}_1, \bm{\Sigma})
			\end{align*}
			则得到 
			\begin{align*}
				p(y) &= \phi^y (1-\phi)^{(1-y)} \\
				p(\bm{x} | y=0) &= \frac{1}{(2\pi)^{n/2} |\bm{\Sigma}|^{1/2}} \exp \left( -\frac{1}{2} (\bm{x}-\bm{\mu}_0)^\intercal \bm{\Sigma}^{-1} (\bm{x}-\bm{\mu}_0) \right) \\
				p(\bm{x} | y=1) &= \frac{1}{(2\pi)^{n/2} |\bm{\Sigma}|^{1/2}} \exp \left( -\frac{1}{2} (\bm{x}-\bm{\mu}_1)^\intercal \bm{\Sigma}^{-1} (\bm{x}-\bm{\mu}_1) \right)
			\end{align*}
			通过 maximum likelihood 
			\begin{align*}
				l(\phi, \bm{\mu}_0, \bm{\mu}_1, \bm\Sigma) &= \log \prod_{i=1}^{m} p(\bm x^{(i)}, y^{(i)}; \phi, \bm{\mu}_0, \bm{\mu}_1, \bm{\Sigma}) \\
				&= \log \prod_{i=1}^{m} p(\bm x^{(i)} | y^{(i)}; \phi, \bm{\mu}_0, \bm{\mu}_1, \bm{\Sigma}) p(y^{(i)}; \phi)
			\end{align*}
			可以解得 
			\begin{align*}
				\phi &= \frac{1}{m} \sum_{i=1}^{m} 1\{y^{(i)}=1\} \\
				\bm{\mu}_0 &= \frac{ \sum_{i=1}^{m} 1\{y^{(i)}=0\} \bm{x}^{(i)}} { \sum_{i=1}^{m} 1\{y^{(i)}=0\} } \\
				\bm{\mu}_1 &= \frac{ \sum_{i=1}^{m} 1\{y^{(i)}=1\} \bm{x}^{(i)}} { \sum_{i=1}^{m} 1\{y^{(i)}=1\} } \\
				\bm{\Sigma} &= \frac{1}{m} \sum_{i=1}^{m} \left( \bm{x}^{(i)} - \mu_{y^{(i)}} \right) \left( \bm{x}^{(i)} - \mu_{y^{(i)}} \right)^\intercal
			\end{align*}
			
		\subsection{GDA and Logistic Regression}
			可以证明,在 GDA 中 $ p(y=1 | \bm{x}; \phi, \bm{\Sigma}, \bm{\mu}_0, \bm{\mu}_1) $ 可以整理为 $ \frac{1}{1+\exp(\bm{\theta}^\intercal \bm{x})} $ 的形式。甚至将 Gaussian distribution 换成任何的 exponential family ,此结论都成立。
			
			事实上,GDA 对数据作了更强的假设,而 logistic regression 中的假设要弱一些。当数据确实符合 Gaussian distribution 时,GDA 的效果比 logistic 
			regression 的效果更好。反之,logistic regression 的效果更好。
			
	\section{Naive Bayes} 
		Naive Bayes 是一种简单的分类算法。例如要解决垃圾邮件的分类问题,我们令 $ y \in {0, 1} $ 表示是否为垃圾邮件,并维护一个字典 ,$ x_i \in {0, 1} $ 表示字典中第 $ i $ 个单词是否出现。
		
		Naive Bayes 算法假设
		\footnote{这个假设被称为 \emph{naive Bayes assumption},虽然很不合理,但很多时候效果不错。}
		:在给定 $ y $ 的情况下,$ x_i $ 条件独立,即
		\begin{align*}
			p(x_1, \ldots, x_n | y) &= p(x_1 | y) p(x_2 | y, x_1) \ldots p(x_n | y, x_1, \ldots, x_{n-1}) \\
			&= \prod_{i=1}^{n} p(x_i | y)
		\end{align*}
		
		设 $ \phi_{i | y=1} = p(x_i=1 | y=1), \phi_{i | y=0} = p(x_i=1 | y=0), \phi_y = p(y=1)$,则通过 maximum likelihood 可得
		\begin{align}
			\phi_{i | y=1} &= \frac {\sum_{j=1}^{m}1\{ x_i^{(j)}=1 \land y^{(j)}=1 \}} {\sum_{j=1}^{m}1\{y^{(j)}=1\}} \\
			\phi_{i | y=0} &\quad\text{类似} \\
			\phi_y &= \frac{\sum_{j=1}^{m}1\{y^{(j)}=1\}}{m}
		\end{align}
		所以由 Bayes 公式得
		\begin{align*}
			p(y=1 | \bm{x}) &= \frac{p(\bm{x} | y=1) p(y=1)}{p(\bm{x} | y=1) p(y=1) + p(\bm{x} | y=0) p(y=0)} \\
			&= \frac { \prod_{i=1}^{m} p(x_i | y=1) p(y=1) } { \prod_{i=1}^{m} p(x_i | y=1) p(y=1) + \prod_{i=1}^{m} p(x_i | y=0) p(y=0) }
		\end{align*}
		实际只需要比较 $ \prod_{i=1}^{m} p(x_i | y=1) p(y=1) \text{和} \prod_{i=1}^{m} p(x_i | y=0) p(y=0) $ 的大小即可。
		
		值得注意的有两点:
		\begin{enumerate}
			\item Naive Bayes 还可以推广到多分类问题。
			\item 一般要加 Laplace smoothing,解决概率为 $ 0 $ 的问题。
		\end{enumerate}
		