\chapter{k-means}
	\section{Algorithm}
		K-means 是一种 unsupervised algorithm。对于给定的 $ k $,将数据集分为 $ k $ 个 cluster。
		\begin{enumerate}
			\item 将 cluster centroids $ \bm{\mu}_1, \ldots, \bm{\mu}_k $ 随机初始化
			\item 循环直到收敛:
			\begin{align}
				\forall i, \quad c^{(i)} &= \argmin_j \| \bm{x}^{(i)} - \bm{\mu}_j \|^2 \label{eq:kmeans-algo1} \\
				\forall j, \quad \bm{\mu}_j &= \frac{\sum_{i=1}^m 1\{c^{(i)}=j\} \bm{x}^{(i)}}{\sum_{i=1}^m 1\{c^{(i)}=j\}} \label{eq:kmeans-algo2}
			\end{align}
		\end{enumerate}
		其中 $ c^{(i)} $ 是每个样本 “对应” 的 cluster 编号。\eqref{eq:kmeans-algo1} 更新编号,\eqref{eq:kmeans-algo2} 更新 cluster centroids 的位置。
	
	\section{Convergence}
		定义 distortion function :
		\begin{equation}
			J(\bm{c}, \bm{\mu}) = \sum_{i=1}^{m} \| \bm{x}^{(i)} - \bm{\mu}_{c^{(i)}} \|^2
		\end{equation}
		易知,k-means 的迭代过程实际上是对 $ J $ 进行 coordinate descent。 $ J $ 单调下降,最终收敛。\footnote{通常 $ \bm{c} $ 和 $ \bm{\mu} $ 也会收敛。}
		
		注意,$ J $ 不是一个凸函数,coordinate descent 可能陷入局部最优值。可以重复运行算法,每次将 cluster centroids 随机初始化为不同的值。