\chapter{Logistic Regression}
	对样本进行分类,标签为 $ y^{(i)} \in \{0, 1\} $。定义 hypothesis function 为
	\footnote{$ g(z) = \frac{1}{1- \exp(-z)} $ 被称为 logistic function。}
	\begin{equation}
		h_{\bm\theta}(\bm{x}) = g(\bm{\theta}^\intercal \bm{x}) = \frac{1}{1 + \exp{({\bm{\theta}}^\intercal \bm{x})}}
	\end{equation}
	由于 $ h_{\bm{\theta}}(\bm{x}) \in (0, 1] $,我们可以将其看做是概率,即
	\begin{align*}
		P(y=1 | \bm{x}; \bm{\theta}) &= h_{\bm{\theta}}(\bm{x}) \\
		P(y=0 | \bm{x}; \bm{\theta}) &= 1 - h_{\bm{\theta}}(\bm{x})
	\end{align*}
	这是一个 Bernoulli 分布,可以简化为 
	\begin{equation}
		P(y | \bm{x}; \bm{\theta}) = (h_{\bm{\theta}}(\bm{x}))^y (1 - h_{\bm{\theta}}(\bm{x}))^{1-y}
	\end{equation}
	
	其 log likelihood 为
	\begin{align*}
		l(\bm{\theta}) &= \sum_{i=1}^{m} \log P(y^{(i)} | \bm{x}^{(i)}; \bm{\theta}) \\
		&= \sum_{i=1}^{m} y^{(i)} \bm{\theta}^\intercal \bm{x} - \log(1+\exp \bm{\theta}^\intercal \bm{x}) 
	\end{align*}
	求导得
	\begin{equation}
		\nabla_{\bm{\theta}} l = \sum_{i=1}^{m} (y^{(i)} - h_{\bm{\theta}}(\bm{x}^{(i)})) \bm{x}^{(i)}
	\end{equation}
	
	综上,stochastic gradient ascent 的每一步迭代为 
	\begin{align*}
		\bm{\theta} &= \bm{\theta} + \alpha (y^{(i)} - h_{\bm{\theta}}(\bm{x}^{(i)})) \bm{x}^{(i)}
	\end{align*}
	其中 $ \alpha $ 为人为选择的步长。
	
	\section{Newton's Method}
		另一种常用的迭代方法是 Newton's method。每次迭代的步骤是
		\begin{equation}
			\bm{\theta} = \bm{\theta} - H^{-1} \nabla_{\bm{\theta}} l(\bm{\theta})
		\end{equation}